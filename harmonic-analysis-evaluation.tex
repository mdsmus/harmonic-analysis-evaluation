\documentclass{article}
\usepackage{ismir}
\usepackage{amsmath}
\usepackage{graphicx}
\usepackage{url}
\usepackage[utf8x]{inputenc}
\usepackage[T1]{fontenc}
\usepackage[english]{babel}
%\usepackage[htt]{hyphenat}
\usepackage{times}
\usepackage{color}

\newcounter{notecounter}

\newcommand{\note}[1]{
  \addtocounter{notecounter}{1}
  \textcolor{red}{[note \arabic{notecounter}: #1]}
}

\title{}
\oneauthor
  {Pedro Kröger, Alexandre Passos, Marcos Sampaio, Givaldo de Cidra}
  {Genos---Computer Music Research Group\\ School of Music
   \\ Federal University of Bahia, Brazil \\
  \url{{pedro.kroger,alexandre.tp,mdsmus,givaldodecidra}@gmail.com}}


\begin{document}
\graphicspath{{figs/}{data/}}
\maketitle

\begin{abstract}
\end{abstract}

\section{Introduction}
\label{sec:introduction}

% metodology
% evaluation
% algoritmos

An analyst performing harmonic analysis of a piece of tonal music aims
to increase his understanding of the piece by extracting meaningful
information from it.  He separates the chords underlying the notes and
finds their root and tonal function.  Fundamentally, harmonic analysis
is just another information retrieval task, and could benefit from
proper treatment as such.

The problem of automated harmonic analysis has been approached from
many directions, mostly mirroring approaches to natural language
processing currently in fashion.  First, Winograd
\cite{winograd:linguistics} and Ulrich \cite{ulrich:analysis}
developed backtracking parsers based of formal grammars and reduction
rules.  After a hiatus they were followed by Maxwell's
\cite{maxwell:expert} rule-based expert system, Temperley and
Sleator's Melisma \cite{temperley.ea:modeling}, based on preference
rules.  These were followed by Pardo and Birmingham's HarmAn
\cite{pardo.ea:automated}, Barthelemy and Bonardi's
\cite{barthelemy.ea:figured} and Taube's workbench
\cite{taube:automatic}, all built around pattern-matching as their
core chord-finding method.  Following these there are Tsui's
\cite{tsui:harmonic} neural network algorithm, Noland and Sandler's
\cite{noland.ea:key} Hidden Markov Model algorithm and Temperley's
\cite{temperley:bayesian} bayesian approach.

All these techniques and methods were described informally, and very
little source code is avaliable for testing.  Some articles, most
notably Barthelemy and Bonardi \cite{barthelemy.ea:figured} and
Temperley and Sleator \cite{temperley.ea:modeling}, don't
provide enough information to reproduce their results reliably.  Every
benchmark found in literature (\cite{pardo.ea:automated,
  barthelemy.ea:figured, tsui:harmonic, taube:automatic,
  illescas.ea:harmonic}) is only based on published examples, and this
possibly difficults a more thorough evaluation of each technique's
main merits and flaws.

Here we frame some known and new approaches to automatic harmonic
analysis into a more systematic Information Retrieval framework, and
evaluate their performance accordingly.  We review Pardo and
Birmingham's HarmAn \cite{pardo.ea:automated}, Maxwell's expert system
\cite{maxwell:expert}, a version of Tsui's neural networks
\cite{tsui:harmonic} and original algorithms based on decision trees
and the k-nearest-neighbours algorithm.
These algorithms are evaluated according to their precision, recall,
bias and variance in respect to the various chord types.  We also
study the effect of incorporating enharmonic information on the many
algorithms. Each algorithm is evaluated against our corpus of 200 Bach
chorales from the Riemenshneider \cite{bach:371} edition.  Some of
these chorales are also used as training data, when applicable.

In this article we first discuss the main characteristic and behavior
of each algorithm. In section \ref{sec:heuristic-algorithms} we
present Pardo et al and Maxwell's algorithms, and discuss possible
enhancements to their original proposals. In section
\ref{sec:stat-algor} we analyse Tsui's neural networks, our decision
trees and a baseline k-nearest-neighbours method. Then, in section
\ref{sec:enharmonic} we study the effect of incorporating enharmonic
information in the inputs for the algorithms. We compare their overall
performance in section \ref{sec:discussion} and state our case in
section \ref{sec:conclusions}


\section{Heuristic algorithms}
\label{sec:heuristic-algorithms}

The defining characteristic of a heuristic algorithm is that its
behavior is based on some form of domain-specific rules manually coded
by an expert in the problem domain.

\section{Machine Learning algorithms}
\label{sec:stat-algor}

Another category of approaches to harmonic analysis is comprised of
the machine learning algorithms. These techniques, originally from
statistics and early AI, work by automatically extracting patterns
from the data. 

\section{Enharmonic Information}
\label{sec:enharmonic}



\section{Discussion}
\label{sec:discussion}



\section{Conclusions}
\label{sec:conclusions}


\bibliographystyle{plain}
\bibliography{strings-short,ismir,programs,coding,harmonic-analysis,dont-have,artifical-inteligence,music-harmony-and-theory,licenses,icmc}

\end{document}

